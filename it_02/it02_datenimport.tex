\documentclass[12pt]{article}
\usepackage[utf8]{inputenc}
\usepackage{mathtools}
\usepackage{amssymb}
\usepackage{amsmath}
\usepackage{amsthm}
\usepackage{array}
\usepackage{setspace}
\usepackage{parskip}
\usepackage{listings}
\title{Datenbanksysteme\\
			 Projekt Iteration 2: Daten import}
\author{Adrian Gruszczynski\\ 
			Pit Ronk\\ %Mathe Bachelor
			Remi Toudic\\ %Mathe Master
		Tutor: Christian Hofmann\\
		Tutorium: Donnerstag 12-14}

\begin{document}
\maketitle
\section*{Aufgabe 1: Datenbankschema erstellen}
\lstinputlisting[language=sql, frame=single, breaklines, numbers=left
]{src/a1.sql}
%Hier noch github Link
\newpage
\section*{Aufgabe 2: Datenbereinigung}
\lstinputlisting[language=python, frame=single, breaklines, numbers=left]{src/clean_data.py}
%hier noch github link
\newpage
\section*{Aufgabe 3: Daten import}
\lstinputlisting[language=python, frame=single, breaklines, numbers=left]{src/import_data.py}
%Again github

\section*{Aufgabe 4: Relationales Modell}

\end{document}

