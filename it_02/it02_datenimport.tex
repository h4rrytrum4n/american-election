\documentclass[12pt]{article}
\usepackage[utf8]{inputenc}
\usepackage{mathtools}
\usepackage{amssymb}
\usepackage{amsmath}
\usepackage{amsthm}
\usepackage{array}
\usepackage{setspace}
\usepackage{parskip}
\usepackage{listings}
\usepackage{hyperref}
\title{Datenbanksysteme\\
			 Projekt Iteration 2: Datenimport}
\author{Adrian Gruszczynski\\ 
			Pit Ronk\\ %Mathe Bachelor
			Remi Toudic\\ %Mathe Master
		Tutor: Toni Draßdo\\
		Tutorium: Donnerstag 12-14}

\begin{document}
\maketitle
\section*{Aufgabe 1: Datenbankschema erstellen}
\lstinputlisting[language=sql, frame=single, breaklines, numbers=left
]{src/a1.sql}
Private repository: \\
\url{https://gitlab.spline.inf.fu-berlin.de/h4rrytrum4n/DBS_PR/blob/master/it_02/src/a1.sql}
\newpage
\section*{Aufgabe 2: Datenbereinigung}
\lstinputlisting[language=python, frame=single, breaklines, numbers=left]{src/clean_data.py}
Private repository: \\
\url{https://gitlab.spline.inf.fu-berlin.de/h4rrytrum4n/DBS_PR/blob/master/it_02/src/clean_data.py}
\newpage
\section*{Aufgabe 3: Daten import}
\lstinputlisting[language=python, frame=single, breaklines, numbers=left]{src/import_data.py}
Private repository: \\
\url{https://gitlab.spline.inf.fu-berlin.de/h4rrytrum4n/DBS_PR/blob/master/it_02/src/import_data.py}

\section*{Aufgabe 4: Webserver}
Für die Webanwendung haben wir uns entschieden Django Framework zu benutzen. 

Django ist ein Python basiertes Webframework das einem Model-View-Present Schema folgt.
Es verfügt über eine integrierte Schnittstelle für die Anbindung von mehreren Datenbanken u.a PostgreSQL.
Das Webframework an sich läuft auf einem Apache Webserver.

Sobald unsere Datenbank eingerichtet wurde, könenn wir mit Installation von Django anfangen. Für verbesserte Flexibilität werden wir die Einrichtung von Django und dessen Abhängigkeiten in einem virtual enviroment einrichten.

Zunächst werden folgende packages installiert:
\begin{lstlisting}[language=sh, frame=single]
sudo apt install python-pip python-dev libpq-dev
sudo pip install virtualenv
\end{lstlisting}
Als nächstes wird ein Projektordner erstellt:
\begin{lstlisting}[language=sh, frame=single]
mkdir django-postgresql
cd django-postgresql
\end{lstlisting}
Nun wird ein virtual enviroment für unseres Django Projekt angelegt und aktiviert:
\begin{lstlisting}[language=sh, frame=single]
virtualenv django-postgresql
source django-postgresql/bin/activate
\end{lstlisting}
Sobald unser virtual enviroment aktiv ist, kan mann mit der Installation von $psycopg2$ anfangen. Das Paket ermöglicht uns den Aufbau einer Verbindung zu unserer Datenbank.
\begin{lstlisting}[language=sh, frame=single]
pip install django psycopg2
\end{lstlisting}
Als nächstes kann das Projekt angelegt werden:
\begin{lstlisting}[language=sh, frame=single]
django-admin.py startproject django-postgresql .
\end{lstlisting}
Wenn unser Projekt erfolgreich angelegt wurde, kann man mit der Konfiguration der Datenbank beginnen. Dafür öffnen wir die folgende Konfigurationsdatei und suchen nach dem $DATABASES$ Abschnitt:
\begin{lstlisting}[language=sh, frame=single]
vim django-postgresql/settings.py
\end{lstlisting}
Die ursprüngliche Konfiguration muss folgendermaßen angepasst werden:
\begin{lstlisting}[language=sh, frame=single]
DATABASES = {
    'default': {
        'ENGINE': 'django.db.backends.postgresql_psycopg2',
        'NAME': 'election',
        'USER': 'postgres',
        'PASSWORD': 'postgres',
        'HOST': 'localhost',
        'PORT': '5432',
    }
}
\end{lstlisting}
Wenn die Konfiguration der Datenbankverbindung erfolgreich gelaufen ist kann man jetzt die bestehende Relationen aus unserer Election Datenbank als Modell importieren:
\begin{lstlisting}[language=sh, frame=single]
python manage.py inspectdb > models.py
pthon manage.py makemigrations
python manage.py migrate
\end{lstlisting}
Nachdem die Struktur für die Datenbank aufgebaut wurde kann man ein administratives Konto erstellen:
\begin{lstlisting}[language=sh, frame=single]
python manage.py createsuperuser
\end{lstlisting}
Zum Schluss starten wir unseren Webserver mit folgendem Befehl:
\begin{lstlisting}[language=sh, frame=single]
python manage.py runserver localhost:8888
\end{lstlisting}




\end{document}

